\documentclass[xcolor=x11names,handout,compress]{beamer}

%% General document %%%%%%%%%%%%%%%%%%%%%%%%%%%%%%%%%%
\usepackage{graphicx}
\graphicspath{{graphics/}}
\usepackage{tikz}
\usepackage{color,colortbl}
\usepackage{framed}
\usepackage{textcomp, setspace} %Needed for customization of ``listings''
% package
\usepackage[procnames]{listings} % to display code; don't forget [fragile]
% option after \begin{frame}
\input{inputs/rgb}
\definecolor{shadecolor}{rgb}{1,.9,.3}

\usepackage [autostyle]{csquotes}
\MakeOuterQuote{"}

\lstset{
    backgroundcolor=\color{shadecolor},
    tabsize=4,
    rulecolor=,
    language=python,
        basicstyle=\ttfamily\setstretch{1},
        upquote=true,
        aboveskip={1.5\baselineskip},
        columns=fixed,
        showstringspaces=false,
        extendedchars=true,
        breaklines=true,
        prebreak = \raisebox{0ex}[0ex][0ex]{\ensuremath{\hookleftarrow}},
        frame=single,
        showtabs=false,
        showspaces=false,
        showstringspaces=false,
        identifierstyle=\ttfamily,
        keywordstyle=\color[rgb]{0,0,1},
        commentstyle=\color[rgb]{0.133,0.545,0.133},
        stringstyle=\color[rgb]{0.627,0.126,0.941},
        numbers=left, 
        numberstyle=\tiny, 
        stepnumber=2, 
        numbersep=5pt
}

%%%%%%%%%%%%%%%%%%%%%%%%%%%%%%%%%%%%%%%%%%%%%%%%%%%%%%

%% Beamer Layout %%%%%%%%%%%%%%%%%%%%%%%%%%%%%%%%%%
\usetheme{Madrid}
\usecolortheme{seahorse}
\useoutertheme[subsection=false,shadow]{miniframes}
\useinnertheme{default}
% \usefonttheme{serif}
\usepackage{palatino}

\setbeamerfont{title like}{shape=\scshape}
\setbeamerfont{frametitle}{shape=\scshape, series = \bfseries}
\setbeamertemplate{frametitle}[default][center]
\setbeamertemplate{headline}{} %suppress headline (navigation pane)

\setbeamertemplate{footline}
{
  \leavevmode%
  \hbox{%
%   \begin{beamercolorbox}[wd=.4\paperwidth,ht=2.25ex,dp=1ex,center]{author in 
% head/foot}%
%     \usebeamerfont{author in head/foot}\insertshortauthor
%   \end{beamercolorbox}%
  \begin{beamercolorbox}[wd=.93\paperwidth,ht=2.25ex,dp=1ex,left]{title in 
head/foot}%
    \usebeamerfont{title in head/foot}\insertshorttitle\hspace*{3em}
  \end{beamercolorbox}}%
  \begin{beamercolorbox}[wd=.07\paperwidth,ht=2.25ex,dp=1ex,center]{}
     \insertframenumber{} / \inserttotalframenumber\hspace*{1ex}
%       \insertframenumber{}
  \end{beamercolorbox}

}
% \setbeamercolor*{lower separation line head}{bg=DeepSkyBlue4} 
% \setbeamercolor*{normal text}{fg=black,bg=white} 
% \setbeamercolor*{alerted text}{fg=red} 
% \setbeamercolor*{example text}{fg=black} 
% \setbeamercolor*{structure}{fg=black}
%  
% \setbeamercolor*{palette tertiary}{fg=black,bg=black!10} 
% \setbeamercolor*{palette quaternary}{fg=black,bg=black!10} 

\renewcommand{\(}{\begin{columns}}
\renewcommand{\)}{\end{columns}}
\newcommand{\<}[1]{\begin{column}{#1}}
\renewcommand{\>}{\end{column}}

\def\signed #1{{\leavevmode\unskip\nobreak\hfil\penalty50\hskip2em
  \hbox{}\nobreak\hfil(#1)%
  \parfillskip=0pt \finalhyphendemerits=0 \endgraf}}

\newsavebox\mybox
\newenvironment{aquote}[1]
  {\savebox\mybox{#1}\begin{quote}}
  {\signed{\usebox\mybox}\end{quote}}

\title{CMEE Masters Week 6: Python week II}
\author{Samraat Pawar}
  
%%%%%%%%%%%%%%%%%%%%%%%%%%%%%%%%%%%%%%%%%%%%%%%%%%%%

\begin{document}

%%%%%%%%%%%%%%%%%%%%%%%%%%%%%%%%%%%%%%%%%%%%%%%%%%%%%%
\begin{frame}[plain]

\title{Fitting Models to Data in Ecology and Evolution}
\vspace{12pt}
\subtitle{CMEE Masters}
\author{
    Samraat Pawar\\
    \vspace{20pt}
  \centering
  \includegraphics[height = .3in]{Imperial_Color1.pdf}
}
 
\titlepage
\end{frame}

%%%%%%%%%%%%%%%%%%%%%%%%%%%%%%%%%%%%%%%%%%%%%%%%%%%%%%
\begin{frame}{What is model selection?}

   \begin{itemize}[<+->]\itemsep12pt
		\item Several competing hypotheses (Mathematical models) are fitted to data and compared statistical theory 
		\item This is an advance over the traditional ``null hypothesis'' approach in Biology
		\item Necessary for developing the advancement of Biology from from an observational and axiomatic discipline to one with general theories.
		\item Necessary for understanding the mechanisms underlying Biological patterns and phenomena   
	 \end{itemize}

\end{frame}

%%%%%%%%%%%%%%%%%%%%%%%%%%%%%%%%%%%%%%%%%%%%%%%%%%%%%%
\begin{frame}{Mechanistic vs. phenomenological models}

   \begin{itemize}\itemsep4pt
		 \item Mechanistic models aim to explain the PROCESSES underlying 
		 observed patterns

		\item Empirical or phenomenological models show relationships 
		between observed data (e.g. population size as a function of 
		temperature or rainfall), but provide no insights into why they are 
		related
	 \end{itemize}

\pause
	
 \begin{center}
	 \includegraphics[width=\textwidth]{Mechanisms.pdf}
 \end{center} 

\end{frame}

%%%%%%%%%%%%%%%%%%%%%%%%%%%%%%%%%%%%%%%%%%%%%%%%%%%%%%
\begin{frame}{What are Mechanisms?}

 \begin{center}
	 \includegraphics[width=.6\textwidth]{Mechanisms.pdf}
 \end{center} 
   
   \begin{itemize}[<+->] \itemsep4pt
			\item Ecological studies often focus on explaining phenomena 
			using somewhat phenomenological models. 
      \item For example, insect invasions, outbreaks and spread {\tiny \url{
      (http://www.sandyliebhold.com/pubs/science_DC1/)}} --- papers in 
      your {\tt Readings} directory.
      \begin{itemize}
				\item Why the cycles?, Why the travelling waves? \pause What 
				mechanisms operate? (budmoth/parasitoid interaction? 
				(budmoth/food quality interaction?) Are these truly mechanisms?
			\end{itemize}
     \item Another example, disease outbreaks (Papers in 
      your {\tt Readings} directory)   
      
      \end{itemize}

\end{frame}

%%%%%%%%%%%%%%%%%%%%%%%%%%%%%%%%%%%%%%%%%%%%%%%%%%%%%%
\begin{frame}{What are Mechanisms?}

   \begin{itemize}[<+->] \itemsep4pt{}
			\item Somewhat subjective! 
			\item For example, the Ricker model can be thought of as 
			mechanistic:
			\begin{equation} 
				N_{t+1} = N_t e^{r\left(1-\frac{N_t}{k}\right)} 
			\end{equation} 
				
			\item What is the mechanism? --- Density dependence through scramble competition 
			(Brannstrom \& Sumpter 2005)

			\item If the Ricker model and another model with contest 
			competition were compared with data --- some would call it mechanistic 
			modelling because one is trying to get at the underlying 
			mechanism, scramble or contest competition
			
			\item But is this REALLY mechanistic? What are $r$ and $k$ really?
	
			\item Many (including yours truly!) now argue that we have not 
			progressed far enough because the first level has been ignored!
   \end{itemize}

\end{frame}

%%%%%%%%%%%%%%%%%%%%%%%%%%%%%%%%%%%%%%%%%%%%%%%%%%%%%%
\begin{frame}{An example of a fundamental mechanism: metabolism}
      
\begin{columns}[c]
  \column{2.1in}
    \pause
      \includegraphics[width=\textwidth]{graphics/Photobacterium.pdf}
  \pause
  \column{2.7in}
    $B = B_0 \boxed {e^{-\frac{E}{kT}}}f(T,T_{pk},E_D)$\\
    \vspace{10pt}
    \raggedright{$T$ = temperature (K)\\
     $k$ = Boltzmann constant (eV K$^{-1}$)}\\
     $E$ = Activation energy (eV)\\
     $T_{pk}$ = Temperature of peak performance\\
     $E_D$ = Deactivation energy (eV)\\
    {\small (J H van't Hoff 1884, S Arrhenius 1889)}
\end{columns}

\begin{itemize}\setlength{\itemindent}{0em}
  \pause
  \item Surely there is more to thermal responses?
    \begin{itemize}\setlength{\itemindent}{-1em}
     \item Oxygen limitation
     \item Complexity of metabolic network
     \item Hormonal regulation
    \end{itemize}
    \pause
  \item \textit{What about alternative models?}\\

\end{itemize}
 
\end{frame}

	%%%%%%%%%%%%%%%%%%%%%%%%%%%%%%%%%%%%%%%%%%%%%%%%%%%%%%
\begin{frame}{Modelling, and fitting models to data: What's the big idea?}

\begin{itemize} \itemsep8pt

	\item If possible, use biological knowledge to construct models
	\item See if the models "agree well" with data
	\item Whichever model "agrees best" is most likely to have the right 
	mechanisms
	\item That's the one that's best for predictions (e.g. population 
	cycles), estimating rates (e.g. population or individual growth rates), etc
	\item Don't use models you already know have the wrong mechanisms just because they are popular!
	\item Phenomenological models often perform better than mechanistic ones

\end{itemize}
  
\end{frame}

%%%%%%%%%%%%%%%%%%%%%%%%%%%%%%%%%%%%%%%%%%%%%%%%%%%%%%

\begin{frame}{Models: How to build them?}

\begin{itemize}\itemsep10pt

	\item It's an art, take practice (look at Levins' paper on the 
	strategy of model building in biology)
	\item Build models one mechanism at a time --- in biology, it means 
	start at the right level of organization! 
	\item Always consider a alternative that is more parsimonous, even if 
	it is phenomenological (the TPC example: Sharpe-Schoolfield, or Polynomial?)! 
\end{itemize}


	
\end{frame}

%%%%%%%%%%%%%%%%%%%%%%%%%%%%%%%%%%%%%%%%%%%%%%%%%%%%%%

\begin{frame}{Models: How to build them?}

\begin{center}
	 \includegraphics[width=\textwidth]{Mechanisms.pdf}
\end{center} 

\begin{itemize}\itemsep10pt

	\item For example, the Boltzmann-Arrhenius model is a good first try 
	describe and uncover mechanisms underlying individual level rates  

	\item The next step would be to include species interactions with  
	temperature dependence of individuals (or go in an evolutionary direction!)

\end{itemize}   



\end{frame}


	% %%%%%%%%%%%%%%%%%%%%%%%%%%%%%%%%%%%%%%%%%%%%%%%%%%%%%%
% \begin{frame}[plain]{}

% \begin{center}

	% {\bf Next: A primer on mechanistic model fitting} \\
	% \pause 
	% \vspace{20pt}
	% But first: A preview of the Miniproject (back to Notes)
	
% \end{center}

% \end{frame}


%%%%%%%%%%%%%%%%%%%%%%%%%%%%%%%%%%%%%%%%%%%%%%%%%%%%%%
\begin{frame}{Fitting models to data}

Two main ways to do it:

\begin{itemize}\itemsep10pt

	\item One-step forecasting and machine learning (appropriate for discrete models) and time 
	series data --- focus in on maximizing ability to predict at the cost of mechanistic insights 

	\item Ensemble fitting (appropriate for full time series or responses)
	\begin{itemize}
	\item Least Squares methods
	\begin{itemize}
	 \item Linear 
	 \item Non-linear
	\end{itemize}
	\item Likelihood based methods 
	\begin{itemize}
	 \item Maximum Likelihood Estimation (MLE) 
	 \item Bayesian
	\end{itemize}

	\end{itemize}
	
\end{itemize}

\end{frame}

%%%%%%%%%%%%%%%%%%%%%%%%%%%%%%%%%%%%%%%%%%%%%%%%%%%%%%
\begin{frame}{Ensemble fitting}

\begin{itemize}
	
\item These include MLE, Bayesian methods, and least squares optimization or fitting. 

\item MLE/Bayesian methods will be taught in Term II

\item But you can go far with least squares methods.

\item Non-linear least squares (NLLS) fitting is a particularly versatile and powerful approach, because many mechanisms in biology and inherently non-linear (Read paper by Bo).  

\end{itemize}

\end{frame}

%%%%%%%%%%%%%%%%%%%%%%%%%%%%%%%%%%%%%%%%%%%%%%%%%%%%%%
\begin{frame}[plain]

 \begin{center}
	 \Large A quick reminder: Hypothesis testing and linear vs. and nonlinear models   
 \end{center} 

\end{frame}
%%%%%%%%%%%%%%%%%%%%%%%%%%%%%%%%%%%%%%%%%%%%%%%%%%%%%%

\begin{frame}{NLLS fitting}

Many of you will use NLLS. Basically, this is how it works:

\begin{enumerate}[<+->] \itemsep4pt
	\item Start with an initial value for each parameter in the model
	\item Generate the curve defined by the initial values
	\item Calculate the residual sum-of-squares (rss)
	\item Adjust the parameters to make the curve come closer to the data 
	points. 
	\begin{itemize}
		
	\item This the tricky part --- you will use the Levenberg-Marquardt 
	algorithm in the {\tt lmfit} package in {\tt python} or the equivalent in {\tt R}
	\end{itemize}
		
		\item Adjust the parameters again so that the curve comes even closer 
	to the points (RSS decreases)
	\item Repeat 4--5
	\item Stop simulations when the adjustments make virtually no 
	difference to the RSS
\end{enumerate}
\end{frame}

%%%%%%%%%%%%%%%%%%%%%%%%%%%%%%%%%%%%%%%%%%%%%%%%%%%%%%

\begin{frame}{NLLS fitting}

Once the algorithm as converged (hopefully -- but you may be surprised 
how well it usually works),
\begin{itemize}[<+->] \itemsep6pt
\item Report the best-fit results, including sums of deviations of the 
data from the final model fit
\item Then compare multiple models (e.g., Schoolfield vs. cubic) 
\end{itemize}

\pause The precise parameter values you obtain will depend in part on 
the initial values chosen and the stopping criteria -- \pause so different 
programs will not always give exactly the same results

% \pause {\tt python} seems to have a better Levenberg-Marqualdt 
% implementation than {\tt R}

\end{frame}

%%%%%%%%%%%%%%%%%%%%%%%%%%%%%%%%%%%%%%%%%%%%%%%%%%%%%%

\begin{frame}{Comparing and selecting models}

\begin{itemize}
	\item It's all about the ``Likelihood'' of a model: \\
	the set of parameter values of the	model ($\theta$) given outcomes ($x$), equals the probability of those 
	observed outcomes given those parameter values, that is,

	$$ \mathcal{L}(\theta |x) = P(x | \theta)$$

	\item The easiest thing to do for you is to use information theory 
	(including AIC and BIC) to compare models. 

	\item Both AIC and BIC use the {\it estimated likelihoods of a model}:\\
	
	AIC: $-2 \ln[\mathcal{L}(\theta |x)] + 2p$\\

	Small sample AIC (AICc): $-2 \ln[\mathcal{L}(\theta |x)] + 2p$\\
	
	BIC (Schwartz criterion): $-2 ln[\mathcal{L}(\theta |x) ] + p \ln(n)$
	
		(where $n = $ sample size, $p$ number of free parameters)
	\item The lower the AIC or BIC, the better. 

\end{itemize}

\end{frame}

%%%%%%%%%%%%%%%%%%%%%%%%%%%%%%%%%%%%%%%%%%%%%%%%%%%%%%

\begin{frame}{Comparing and selecting models}

This is how you calculate AIC and BIC (using python syntax): 

\pause
\begin{itemize}
\small
	\item residuals = Observations - Predictions
	\item rss = sum(residuals ** 2) 
	\item Then, AIC is n * log((2 * pi) / n) + n + 2 + n * log(rss) + 2 * p \\
		({\it note n and p!})
	\item And BIC is n + n * log(2 * pi) + n * log(rss / n) + (log(n)) * (p + 1)

	\item That is, 	$ \mathcal{L}(\theta |x) = -\frac{n}{2/ln(RSS/n)}$
	\item For both AIC and BIC, If model {\bf A} has AIC lower by 2-3 or 
	more than model {\bf B}, it's better --- Differences of less than 2-3 
	don't really matter


\end{itemize}

    \pause
Also note that:
\begin{itemize}
\small
	\item R$^{2}$ = 1 - (rss$/$tss), where tss is total sum of squares: \\
tss = sum((Observations - mean(Predictions)) ** 2)\\
(a useful measure of goodness of fit -- you should report it)

\end{itemize}

\end{frame}

%%%%%%%%%%%%%%%%%%%%%%%%%%%%%%%%%%%%%%%%%%%%%%%%%%%%%%

\begin{frame}{Comparing and selecting models}

\begin{itemize} \itemsep12pt
	\item Likelihood-Ratio test (LRT) and Adjusted $R^2$ are two other options. 
	\item There are functions in {\tt R} that allow you to perform model selection/simplification {\it for linear least squares model fitting}.  
\end{itemize}

\end{frame}

%%%%%%%%%%%%%%%%%%%%%%%%%%%%%%%%%%%%%%%%%%%%%%%%%%%%%%
\begin{frame}{Readings}

\begin{itemize}

\item Levins, R. (1966) The strategy of model building in population 
biology. Am. Sci. 54, 421--431.  

\item Johnson, J. B. \& Omland, K. S. (2004) Model selection in ecology 
and evolution. Trends Ecol. Evol. 19, 101--108. 

\item Bolker, B. M. et al.  (2013) Strategies for fitting nonlinear ecological models in R, AD Model Builder, and BUGS. Methods Ecol. Evol. 4, 501--512 .

\item Some illustrative examples of (non-linear) model fitting to 
ecological/evolutionary data 
\url{https://groups.nceas.ucsb.edu/non-linear-modeling/projects} 

\item Additional readings at the end of Miniproject Chapter of your CMEE Notes 
 
\end{itemize}
\end{frame}

\end{document}
